\section{Analysis}\label{analysis}

\subsection{newwin}\label{TCLnewwin}

Syntax:

{\tt newwin} {\em name}

Creates a new toplevel window for a TCL widget, with Tk identifier
{\em name}. {\em name} must start with a full stop.

\subsection{Reduction Functions}

Syntax:

{\tt max} {\em tcllist}\\
{\tt min} {\em tcllist}\\
{\tt av} {\em tcllist}\\

The maximum, minimum and average of a TCL list.

\subsection{Palette Variable}\label{palette}

This variable is used by both the {\tt display\_stub} and {\tt
  connect\_stub} routines to define a palette of colours for colouring
  the species in the instruments. If the TCL variable {\tt palette} is
  assigned a list of X-windows colours (on many systems, a list of
  such colours is found in {\tt /usr/lib/X11/rgb.txt}),
  then the palette class can be used like an array within C++,
  returning the colour name as a string:
\begin{verbatim}
palette[i]
\end{verbatim}
returns the \verb|i%n|th colour, where {\tt n} is the number of
colours in the palette list.

\subsection{trap}\label{trap}

Syntax:

{\tt trap} {\em signal} {\em TCL command}\\
{\tt trapabort}\\
{\tt trapabort off}

{\tt trap}\index{trap} allows a specific handler to
be called when a signal is received. {\em signal} may be a numeric
value, or a symbolic value (eg {\tt TERM}).

{\tt trapabort}\index{trapabort} captures a few fatal signals such as
{\tt SIGABRT} or {\tt SIGSEGV}, and calls the tcl++ {\tt error()}
function. This gracefully returns control back to the TCL
interpreter. {\tt trapabort off} reverses this effect. It is not
desirable to have {\tt trapabort} set when running under a debugger
such as gdb, or in batch mode, but it is desirable when using the
system in an exploratory GUI mode.





